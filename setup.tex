% !TeX root = ./main.tex

% 表格加脚注
\usepackage{threeparttable}

% 表格中支持跨行
\usepackage{multirow}

% 固定宽度的表格。
% \usepackage{tabularx}

% 跨页表格
\usepackage{longtable}

% 算法
\usepackage{algorithm}
\usepackage{algorithmic}

% 量和单位
\usepackage{siunitx}

% 参考文献使用 BibTeX + natbib 宏包
% 顺序编码制
\usepackage[sort]{natbib}
\bibliographystyle{thuthesis-numeric}

% 著者-出版年制
% \usepackage{natbib}
% \bibliographystyle{thuthesis-author-year}

% 本科生参考文献的著录格式
% \usepackage[sort]{natbib}
% \bibliographystyle{thuthesis-bachelor}

% 参考文献使用 BibLaTeX 宏包
% \usepackage[style=thuthesis-numeric]{biblatex}
% \usepackage[style=thuthesis-author-year]{biblatex}
% \usepackage[style=apa]{biblatex}
% \usepackage[style=mla-new]{biblatex}
% 声明 BibLaTeX 的数据库
% \addbibresource{ref/refs.bib}

% 数学命令
\makeatletter
\newcommand\dif{%  % 微分符号
  \mathop{}\!%
  \ifthu@math@style@TeX
    d%
  \else
    \mathrm{d}%
  \fi
}
\makeatother

% hyperref 宏包在最后调用
\usepackage{hyperref}


% // TODO 取消PDF压缩,加快速度,最终版本生成的时候最好把这句话注释掉
% \special{dvipdfmx:config z 0}


% 以下是自定义部分
\usepackage{ctex}

\usepackage{geometry}
\geometry{left=2.54cm,right=2.54cm,top=3.18cm,bottom=3.18cm}

% 字体设置
\usepackage{fontspec}

% 数学相关:
\usepackage{amsmath,amssymb,amsfonts}

% \usepackage{mathrsfs} % 提供数学花体\mathsrc{},但在unicode-math作用下会被\mathcal{}覆盖

% 修改被unicode-math影响的\mathbb{}命令
\let\mathbbalt\mathbb
\usepackage{unicode-math}
\let\mathbb\mathbbalt  % 开启后使用衬线版本的原始\mathbb{};关闭后使用无衬线版本的\mathbb{},但是可以使用\mathbbalt{}输出衬线版本

% 允许多行公式换页
\allowdisplaybreaks[4]

% 浮动体宏包,自动调整插入图表的位置[!htbp]
\usepackage{float}

% 控制题目/说明文字的样式
\usepackage{caption}

% 三线表
\usepackage{booktabs}

\usepackage{graphicx}

% 定义所有的图片文件在 figures 子目录下
\graphicspath{{figures/}}

% 插入偏左或右的图片
\usepackage{wrapfig}


% 自定义命令
\newcommand{\md}{\mathrm{d}}  % 微分算符d
\newcommand{\me}{\mathrm{e}}  % 自然常数e
\newcommand{\mi}{\mathrm{i}}  % 虚数单位i
\newcommand{\mdl}{\updelta}  % 变分符号delta
\newcommand{\mpi}{\uppi}  % 圆周率pi
\newcommand{\mdd}[2]{\frac{\mathrm{d} #1}{\mathrm{d} #2}} % 导数
\newcommand{\re}[1]{\mathrm{Re}(#1)}  % 实部 Re
\newcommand{\im}[1]{\mathrm{Im}(#1)}  % 虚部 Im

\newcommand{\myfrac}[2]{\left.#1\middle/#2\right.}  % 行内分式,使用更大的斜线

% 定理类环境宏包
\usepackage{amsthm}
% 也可以使用 ntheorem
% \usepackage[amsmath,thmmarks,hyperref]{ntheorem}
\newtheorem{theorem}{\indent 定理}[section]
\newtheorem{lemma}[theorem]{\indent 引理}
\newtheorem{proposition}[theorem]{\indent 命题}
\newtheorem{corollary}[theorem]{\indent 推论}
\newtheorem{definition}{\indent 定义}[section]
\newtheorem{example}{\indent 例}[section]
\newtheorem{remark}{\indent 注}[section]
\newenvironment{solution}{\begin{proof}[\indent\textbf{解}]}{\end{proof}}
\renewcommand{\proofname}{\indent\textbf{证明}}
% 在windows环境下textbf是黑体,Linux下是加粗宋体,可替换为textsf。

% 插入代码
\usepackage{listings}
\usepackage[dvipsnames]{xcolor}
\lstset{
  % language=c, % 设置语言,为空默认所有;需要在使用lstlisting环境时指定语言,否则无法识别关键字
  basicstyle=\ttfamily, % 设置字体族
  breaklines=true, % 自动换行
  keywordstyle=\bfseries\color{NavyBlue}, % 设置关键字为粗体,颜色为 NavyBlue
  morekeywords={}, % 设置更多的关键字,用逗号分隔
  emph={self}, % 指定强调词,如果有多个,用逗号隔开
  emphstyle=\bfseries\color{Rhodamine}, % 强调词样式设置
  commentstyle=\itshape\color{black!50!white}, % 设置注释样式,斜体,浅灰色
  stringstyle=\bfseries\color{PineGreen!90!black}, % 设置字符串样式
  columns=flexible,
  numbers=left, % 显示行号在左边
  numbersep=2em, % 设置行号的具体位置
  numberstyle=\footnotesize, % 缩小行号
  frame=single, % 边框
  framesep=1em % 设置代码与边框的距离
}


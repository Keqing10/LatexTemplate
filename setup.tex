% !TeX root = ./main.tex

% 定理类环境宏包
\usepackage{amsthm}
% 也可以使用 ntheorem
% \usepackage[amsmath,thmmarks,hyperref]{ntheorem}

% 表格加脚注
\usepackage{threeparttable}

% 表格中支持跨行
\usepackage{multirow}

% 固定宽度的表格。
% \usepackage{tabularx}

% 跨页表格
\usepackage{longtable}

% 算法
\usepackage{algorithm}
\usepackage{algorithmic}

% 量和单位
\usepackage{siunitx}

% 参考文献使用 BibTeX + natbib 宏包
% 顺序编码制
\usepackage[sort]{natbib}
\bibliographystyle{thuthesis-numeric}

% 著者-出版年制
% \usepackage{natbib}
% \bibliographystyle{thuthesis-author-year}

% 本科生参考文献的著录格式
% \usepackage[sort]{natbib}
% \bibliographystyle{thuthesis-bachelor}

% 参考文献使用 BibLaTeX 宏包
% \usepackage[style=thuthesis-numeric]{biblatex}
% \usepackage[style=thuthesis-author-year]{biblatex}
% \usepackage[style=apa]{biblatex}
% \usepackage[style=mla-new]{biblatex}
% 声明 BibLaTeX 的数据库
% \addbibresource{ref/refs.bib}

% 数学命令
\makeatletter
\newcommand\dif{%  % 微分符号
  \mathop{}\!%
  \ifthu@math@style@TeX
    d%
  \else
    \mathrm{d}%
  \fi
}
\makeatother

% hyperref 宏包在最后调用
\usepackage{hyperref}

% 以下是自定义部分
\usepackage{ctex,amsmath,graphicx,float,booktabs,amsfonts,mathrsfs,amssymb,caption,xcolor,listings,fontspec,multirow,unicode-math,geometry}
\geometry{left=2.54cm,right=2.54cm,top=3.18cm,bottom=3.18cm}

\newcommand{\md}{\mathrm{d}}  % 微分算符d
\newcommand{\me}{\mathrm{e}}  % 自然常数e
\newcommand{\mi}{\mathrm{i}}  % 虚数单位i
\newcommand{\mdl}{\updelta}  % 变分符号delta
\newcommand{\mpi}{\uppi}  % 圆周率pi

% // TODO 取消PDF压缩,加快速度,最终版本生成的时候最好把这句话注释掉
\special{dvipdfmx:config z 0}

% 定义所有的图片文件在 figures 子目录下
\graphicspath{{figures/}}

% !TeX root = ../main.tex

\chapter{图表示例}

\section{插图}

$$\int_a^b\!\alpha\beta\pi\delta\mathrm{e}^x\,\mathrm{d}x \Delta$$

$$\int_a^b\!\upalpha\upbeta\mpi\mdl\me^x\,\md x \upDelta$$
$$\int_a^b e^x dx$$
\[\cos x+y\]

\begin{figure}[!htbp]
    \centering
    \includegraphics[width=0.5\linewidth]{264498559_cover.png}
    \caption{Enter Caption}
    \label{fig:enter-label}
\end{figure}


\begin{figure}
  \centering
  \includegraphics[width=0.5\linewidth]{example-image-a.pdf}
  \caption*{国外的期刊习惯将图表的标题和说明文字写成一段,需要改写为标题只含图表的名称,其他说明文字以注释方式写在图表下方,或者写在正文中。}
  \caption{示例图片标题}
  \label{fig:example}
\end{figure}


\section{表格}

表应具有自明性。为使表格简洁易读,尽可能采用三线表,如表~\ref{tab:three-line}。
三条线可以使用  宏包提供的命令生成。

\begin{table}
  \centering
  \caption{三线表示例}
  \begin{tabular}{ll}
    \toprule
    文件名          & 描述                         \\
    \midrule
    thuthesis.dtx   & 模板的源文件,包括文档和注释 \\
    thuthesis.cls   & 模板文件                     \\
    thuthesis-*.bst & BibTeX 参考文献表样式文件    \\
    \bottomrule
  \end{tabular}
  \label{tab:three-line}
\end{table}

表格如果有附注,尤其是需要在表格中进行标注时,可以使用 宏包。
研究生要求使用英文小写字母 a、b、c……顺序编号,本科生使用圈码 ①、②、③……编号。

\begin{table}
  \centering
  \begin{threeparttable}[c]
    \caption{带附注的表格示例}
    \label{tab:three-part-table}
    \begin{tabular}{ll}
      \toprule
      文件名                 & 描述                         \\
      \midrule
      thuthesis.dtx\tnote{a} & 模板的源文件,包括文档和注释 \\
      thuthesis.cls\tnote{b} & 模板文件                     \\
      thuthesis-*.bst        & BibTeX 参考文献表样式文件    \\
      \bottomrule
    \end{tabular}
    \begin{tablenotes}
      \item [a] 可以通过 xelatex 编译生成模板的使用说明文档;
        使用 xetex 编译 时则会从  中去除掉文档和注释,得到精简的  文件。
      \item [b] 更新模板时,一定要记得编译生成  文件,否则编译论文时载入的依然是旧版的模板。
    \end{tablenotes}
  \end{threeparttable}
\end{table}

如某个表需要转页接排,可以使用 宏包,需要在随后的各页上重复表的编号。
编号后跟表题(可省略)和“(续)”,置于表上方。续表均应重复表头。

\begin{longtable}{cccc}
    \caption{跨页长表格的表题}
    \label{tab:longtable} \\
    \toprule
    表头 1 & 表头 2 & 表头 3 & 表头 4 \\
    \midrule
  \endfirsthead
    \caption*{续表~\thetable\quad 跨页长表格的表题} \\
    \toprule
    表头 1 & 表头 2 & 表头 3 & 表头 4 \\
    \midrule
  \endhead
    \bottomrule
  \endfoot
  Row 1  & & & \\
  Row 2  & & & \\
  Row 3  & & & \\
  Row 4  & & & \\
  Row 5  & & & \\
  Row 6  & & & \\
  Row 7  & & & \\
  Row 8  & & & \\
  Row 9  & & & \\
  Row 10 & & & \\
\end{longtable}



\section{算法}

算法环境可以使用 或者宏包。

\renewcommand{\algorithmicrequire}{\textbf{输入:}\unskip}
\renewcommand{\algorithmicensure}{\textbf{输出:}\unskip}

\begin{algorithm}
  \caption{Calculate $y = x^n$}
  \label{alg1}
  \small
  \begin{algorithmic}
    \REQUIRE $n \geq 0$
    \ENSURE $y = x^n$

    \STATE $y \leftarrow 1$
    \STATE $X \leftarrow x$
    \STATE $N \leftarrow n$

    \WHILE{$N \neq 0$}
      \IF{$N$ is even}
        \STATE $X \leftarrow X \times X$
        \STATE $N \leftarrow N / 2$
      \ELSE[$N$ is odd]
        \STATE $y \leftarrow y \times X$
        \STATE $N \leftarrow N - 1$
      \ENDIF
    \ENDWHILE
  \end{algorithmic}
\end{algorithm}

\section{多栏公式}
\begin{flalign*}
  &(1) \frac15\me^{5t}+C  
  &&(2) -\frac{(3-2x)^4}{8}+C 
  &&(3) -\frac12\ln|1-2x|+C  &\\
  &(4) -\frac12(2-3x)^{\frac23}+C 
  &&(5) -\frac{\cos ax}{a}-b\me^{\frac{x}{b}}+C 
  &&(6) -2\cos\sqrt{t}+C  \\
  &(7)-\frac12\me^{-x^2}+C 
  &&(8) \frac12 \sin(x^2)+C 
  &&(9) -\frac{\sqrt{2-3x^2}}{3}+C
\end{flalign*}
\begin{flalign*}
  &(1) \frac15\me^{5t}+C  
  &&(2) -\frac{(3-2x)^4}{8}+C &\\
  &(3) -\frac12\ln|1-2x|+C  
  &&(4) -\frac12(2-3x)^{\frac23}+C \\
  &(5) -\frac{\cos ax}{a}-b\me^{\frac{x}{b}}+C 
  &&(6) -2\cos\sqrt{t}+C  \\
  &(7)-\frac12\me^{-x^2}+C 
  &&(8) \frac12 \sin(x^2)+C \\
  &(9) -\frac{\sqrt{2-3x^2}}{3}+C
\end{flalign*}


\begin{proof}
  这是一个证明。
\end{proof}

\begin{solution}
  这是一个解。
\end{solution}

\begin{theorem}[定理名字]
  这是一个定理。
\end{theorem}

\begin{definition}
  [定义名字]
  这是一个定义。
\end{definition}
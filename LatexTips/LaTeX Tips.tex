\documentclass{article}
\usepackage{ctex,amsmath,graphicx,float,booktabs,amsfonts,mathrsfs,amssymb,caption,xcolor,listings,fontspec,multirow,unicode-math,geometry}
\geometry{left=2.54cm,right=2.54cm,top=3.18cm,bottom=3.18cm}

\setmonofont[Mapping={}]{Consolas}
\setCJKmonofont{新宋体}
\definecolor{mygreen}{rgb}{0,0.6,0}
\definecolor{mygray}{rgb}{0.5,0.5,0.5}
\definecolor{mymauve}{rgb}{0.58,0,0.82}
\lstset{ %
	numbers=left,
	numberstyle=\tiny,
	backgroundcolor=\color{white},   % choose the background color
	basicstyle=\footnotesize\ttfamily,        % size of fonts used for the code
	columns=fullflexible,
	breaklines=true,                 % automatic line breaking only at whitespace
	captionpos=b,                    % sets the caption-position to bottom
	tabsize=4,
	commentstyle=\color{mygreen},    % comment style
	escapeinside={\%*}{*)},          % if you want to add LaTeX within your code
	keywordstyle=\color{blue},       % keyword style
	stringstyle=\color{mymauve}\ttfamily,     % string literal style
	frame=single,
	rulesepcolor=\color{red!20!green!20!blue!20},
	% identifierstyle=\color{red},
	language=c++,
}

% \newcommand{\md}{\mathop{}\negthinspace\mathrm{d}}
% \newcommand{\mi}{\mathop{}\negthinspace\mathrm{i}}
% \newcommand{\me}{\mathop{}\negthinspace\mathrm{e}}

\title{\LaTeX{} Tips}
\date{}
\author{}
\begin{document}
\maketitle
\section{\LaTeX 规范}

\subsection{数学}
\subsubsection{}
\textbf{希腊字母}

圆周率pi:\verb|\uppi|~$\uppi$。

正体的希腊字母使用 \verb|\upalpha\upbeta\upOmega\upomega|~$\upalpha\upbeta\upOmega\upomega$~$\alpha\beta\Omega\omega$等,要使用\verb|\unicode-math|宏包(或\verb|\upgreek|宏包,但会改变字体)。

加粗的希腊字母使用 \verb|\boldsymbol{A\alpha\Omega\omega}|~$\boldsymbol{A\alpha\Omega\omega}~A\alpha\Omega\omega$,会保持斜体。\verb|\mathbf{}|不能识别。

\subsubsection{}

LaTeX 默认的实部和虚部函数\verb|\Re|~$\Re$ 和\verb|\Im|~$\Im$ 是 Fraktur 体的字母“R”和“I”,但是国标要求使用罗马体的“Re”和“Im”,设置的方法是
\begin{lstlisting}
	\renewcommand{\Re}{\operatorname{Re}}
\end{lstlisting}
\renewcommand{\Re}{\operatorname{Re}}
实部$\Re z$。

虚数单位$\mathrm{i}\symup{i}$。

% 虚数单位:`$\symup{i}$`,不同于 $\mathrm{i}$。

\subsubsection{}

拉丁字母的斜体加粗使用:\verb|\boldsymbol{}|;
\verb|\mathbf{}|会变成正体加粗。

\subsubsection{}

水平分式的“/”调整大小:\verb|\left. \#1 \middle/ \#2 \right.$|。
也可以定义:
\begin{lstlisting}
\newcommand{\myfrac}[2]{\left.#1\middle/#2\right.}
\end{lstlisting}
\subsubsection{}

积分式调整间距:\verb|$\int_{a}^{b}\! f(x) \,\mathrm{d}x$|。

$$
\int_{a}^{b}\! f(x) \,\mathrm{d}x
$$



\subsection{中文}

\textbf{引用}
要加空格:`图~\verb|\ref{label}|`。
“~” = “\ ”



\subsubsection{}
\subsection{其他格式}
加入一个水平线:
\begin{lstlisting}
	\noindent\rule{\textwidth}{1pt}
\end{lstlisting}


\noindent\rule{\textwidth}{1pt}

\subsubsection{}
\textbf{空格}
\begin{itemize}
	\item \verb|a\qquad b|~$a\qquad b$
	\item \verb|a\quad b|~$a\quad b$
	\item \verb|a\enspace b|~$a\enspace b$
	\item \verb|a\; b|~$a\; b$
	\item \verb|a\: b|~$a\: b$
	\item \verb|a\, b|~$a\, b$
	\item \verb|a\! b|~$a\! b$
\end{itemize}



\section{其他}
中文(shift+6)…… 中间($\sim$)······

cdots $\cdots$

ldots $\ldots$


\end{document}
